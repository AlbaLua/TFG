%%%%%%%%%%%%%%%%%%%%%%%%%%%%%%%%%%%%%%%%%%%%%%%%%%%%%%%%%%%%%%%%%%%%%
% $Id: preamble.tex, v1.0 misb Exp $
%%%%%%%%%%%%%%%%%%%%%%%%%%%%%%%%%%%%%%%%%%%%%%%%%%%%%%%%%%%%%%%%%%%%%
%%%Para separar correctamente las palabras de multitud de idiomas%%%
\usepackage[spanish,es-noshorthands]{babel}
%Este paquete permite poner acentos directamente. Tipo de codificación
\usepackage[utf8]{inputenc} 

%\usepackage[latin1]{inputenc}
\usepackage[T1]{fontenc}
\usepackage{listings}
\usepackage{multicol}
%\usepackage{ifthen}
\usepackage{pdflscape}%\usepackage{lscape}
\usepackage{tocbibind}
\usepackage{multirow}
\usepackage{float}
\usepackage{eurosym} % to create euro symbol
\usepackage{textcomp} %para poner el +/-

% Incluimos las subsubsections en la tabla de contenidos
\setcounter{tocdepth}{4}
\setcounter{secnumdepth}{3}

% Encabezados quitados en hojas en blanco
\makeatletter
  \def\cleardoublepage{\clearpage\if@twoside \ifodd\c@page\else
  \vspace*{\fill}
    \thispagestyle{empty}
    \newpage
    \if@twocolumn\hbox{}\newpage\fi\fi\fi}
\makeatother


%\usepackage{lscape}
%Better support for graphics. Builds upon the graphics package
\usepackage{graphicx}%[dvips]
%\usepackage[pdftex]{graphicx}
\usepackage{latexsym}
\usepackage{epsfig}
%Paquete para la cabecera de las páginas (nombre, capítulo)
\usepackage{fancyhdr}
\usepackage{color}
%\usepackage{wrapfig}
%\usepackage{subfigure} %\usepackage{subfig}
\usepackage{subcaption}
%\usepackage{epstopdf}
%\usepackage{here}

%%% HIPERVÍNCULOS %%%
%\usepackage[dvipdfm,colorlinks=true,linkcolor=blue]{hyperref}
\usepackage[colorlinks=true,linkcolor=blue]{hyperref}
\hypersetup{ % Cambiar colores 
  colorlinks,
  citecolor=green,
  linkcolor=blue,
  urlcolor= cyan
}

%%% Macros AMS %%%
\usepackage{amsmath}
\usepackage{amsthm}						%%%Macros AMS para teoremas%%%
\usepackage{amsfonts}					%%%Permite usar fuentes AMS%%%
\usepackage{amssymb}					%%%Para usar simbolos AMS%%%

%\usepackage{picins}
\usepackage{boxedminipage}
\usepackage{shadow}
\usepackage{titlesec}
\usepackage{curves}
\usepackage{rotating}
\usepackage{calc}
%\usepackage[below]{placeins}		%%% Situar figuras hasta (incluida) la página donde comienza una nueva sección si en la misma aparece algo de texto de la sección precedente.
\usepackage[section]{placeins}		%%% Figuras dentro de su sección
%\input{Diagrama-Bloques}

%\setcounter{tocdepth}{3}
\spanishdecimal{.}

%--------------------------------------------------------------------------
% Definiciones de estilo de curvas
%--------------------------------------------------------------------------
\definecolor{miazul}{rgb}{0,0,0.9}
\definecolor{mirojo}{rgb}{0.9,0,0}
\def\miestiloa{\curvedashes[1.0mm]{0,0}}
\def\miestilob{\put(0,0){\makebox{}}\color{miazul}\curvedashes[1.0mm]{0}}
\def\miestiloc{\put(0,0){\makebox{}}\color{mirojo}\curvedashes[1.0mm]{0}}
\def\miestilod{\curvedashes[1.0mm]{0.5,0.4}}
%---- Otras posibilidades --------------------------------------------------
%\def\miestilob{\curvedashes[1.0mm]{0.5,0.4}}
%\def\miestiloc{\curvedashes[1.0mm]{0.5,3}}
%---------------------------------------------------------------------------
% Tamaño de las etiquetas en las figuras generadas con el programa de Matlab
%---------------------------------------------------------------------------
\def\tamanoetiquetas{\footnotesize}
\curvewarnfalse   % Evita warning por tramos rectos en una curva
%\curvewarntrue 
%----------------------------------------------------------------------------

\usepackage{anysize} %Soporte para el comando \marginsize
\marginsize{3.5cm}{2.5cm}{2.5cm}{2.5cm}	%Permite manejar los márgenes de forma sencilla