%chapter introduce un nuevo capítulo
\chapter{Resumen}

Actualmente las redes sociales están en auge y en continuo crecimiento. Cada vez son más los usuarios que pertenecen a estas comunidades y hacen uso de ellas en su vida cotidiana. Revisar tus redes sociales en determinados momentos del día es un hábito que se ha creado de la necesidad de estar conectados. Este fenómeno ha aumentado con la expansión de los terminales móviles, que hacen posible tener al alcance de la mano su acceso en cualquier momento y lugar. 

En este proyecto se presenta una aplicación centrada en una de las redes sociales de mayor extensión e importancia a día de hoy, Facebook. Se estima que más de un 80\% de la población tiene una cuenta en esta red social. Por eso, gracias a su gran extensión, es una fuente de información de la que se pueden realizar profundos análisis para describir patrones de los perfiles existentes.

La aplicación lleva a cabo un estudio de las páginas de Facebook, ofreciendo un análisis tanto de la información de las propias páginas como un perfilado de los usuarios que siguen dichas páginas. 

El objetivo principal de la aplicación es dar acceso a aquellos individuos, usuarios o no de Facebook, que estén interesados en obtener información de un conjunto de páginas de un mismo sector elegido previamente. 

El estudio se centra en un apartado específico dentro de las páginas de Facebook, llamado \textit{Post to Page}. Esta sección la incluyen muchas empresas, marcas o asociaciones en sus páginas, con el fin de dar libertad de expresión a sus usuarios de una forma directa, facilitando a la entidad establecer una relación más cercana e interactiva con sus seguidores.
 
En este apartado los usuarios suelen expresar su agradecimiento o descontento con algún servicio ofrecido, también escriben mensajes de apoyo ante algún acontecimiento relacionado con la página o meras críticas que resultan ser constructivas para la empresa propietaria de la página en Facebook, ayudándoles a mejorar. Por este motivo, este trabajo de fin de grado se ha centrado en esta sección y ofrece un análisis de datos basado en la atención que reciben los usuarios por parte de las páginas ante sus comentarios y la popularidad de dichos comentarios. Por otra parte, también realiza un perfilado de los usuarios que más comentan en las páginas, con el fin de situarlas en un marco sociocultural.  

\textbf{Palabras clave:} aplicación, crawler, Facebook, red social, minería de datos. 

\chapter{Abstract}

Currently social networks are increasing very quickly. Each day there are more users that belong to those communities and use them regularly. Checking notifications in social networks several times a day has become a habit that generates the need to be connected to the net. This fact has increased with the  expansion of mobile terminals that make the access possible anytime, anywhere.

This work introduces an application focused in one of the biggest and most important social network nowadays, Facebook. It has estimated more than 80\% of population has an account in this social network. Thanks to its big extension, it is a source of information from which can be obtained some  data analysis and user patterns.

The proposed application performs a study of Facebook pages, providing an analysis of both the information of those pages, and a profile of the users following them.

The main objective of the application is to provide access to individuals, whether they are or not Facebook users, who are interested in obtaining a study of a set of pages in the same sector preselect by himself.

The study is focused on a particular section of the Facebook page called \textit{Post to page}. This section is included by many companies, brands or associations in their websites in order to give free expression to their users on a direct way making easy an oncoming with their followers.

In this section, users often express their appreciation or dissatisfaction with any service offered. It is also used to show support to any related event in the page or constructive criticism for the company that owns the page. For this reason, this work is focused in that section, providing data analysis based on the attention to comments users by the pages and the popularity of those. In addition, it is also performed a profile of the users that make more comments in the pages in order to place them in a socio-cultural context.

\textbf{Keywords:} application, crawler, Facebook, social network, data mining.