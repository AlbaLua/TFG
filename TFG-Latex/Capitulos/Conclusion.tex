\chapter{Conclusiones}

Se presentan a continuación las conclusiones derivadas de la realización de este trabajo fin de grado. A su vez, se detallarán posibles mejoras aplicables para trabajos futuros.

\section{Conclusión}

En este trabajo fin de grado se ha desarrollado una aplicación web basada en el análisis estadístico de las páginas de Facebook. Esta aplicación ofrece al usuario la posibilidad de elegir las páginas que desea analizar y las características del estudio a realizar. Como se tratan de datos personales, el acceso a la aplicación está protegido de forma que sólo los usuarios registrados en la aplicación podrán tener acceso. Además de que sólo podrán ver sus datos y no los de ningún otro usuario. 

Gracias a la aplicación desarrollada, cualquier usuario puede conocer las características de una página de Facebook de manera automática, sin necesidad de recorrer cada una de las páginas e intuir en función de la actividad de la página cuáles son sus cualidades, o si hay algún factor que las haga distintivas respecto a otras. Se considera una aplicación pionera en ofrecer estos datos a usuarios externos, ya que, actualmente, no existe ninguna aplicación conocida que sea capaz de realizar esto con los datos relacionados de Facebook. 

Se trata de una aplicación de gran utilidad para las empresas, si lo que desean es ampliar su mercado mediante las redes sociales, distinguiéndose del resto.
Los datos proporcionados por la aplicación, pueden servir para aplicar tareas de marketing a raíz dichos datos. Mediante dichos datos se puede conocer cómo actúan sus mayores competidores y mejorar los aspectos que consideren oportunos.

Para la realización de las funcionalidades de esta aplicación se ha tenido que desarrollar un programa capaz de recoger y almacenar todos los datos necesarios de las páginas de Facebook a través de su API, además de la realización de un \textit{crawler} capaz de recoger todos los datos de los usuarios de Facebook.

Destacar también un mayor grado de dificultad de este trabajo manejando dos tipos de bases de datos distintos, lo que ha supuesto un reto más. Además de tener que aplicarse diferentes tecnologías para su realización.

Concluyendo, a pesar de la complejidad de alguna de las tareas realizadas, los objetivos del presente proyecto han sido concluidos por completo. Además el trabajo realizado ha servido de gran aprendizaje para la autora del mismo. Por lo tanto se puede concluir el trabajo con una valoración muy positiva del mismo.

\section{Desarrollos futuros}
Para concluir con la documentación de este trabajo fin de grado se proponen varias mejoras posibles para un trabajo futuro.
\begin{itemize}
\item \textbf{Mejorar la accesibilidad de los usuarios:} Integrar la aplicación en diferentes plataformas como \textit{Android} e \textit{iOS}, para permitir su accesibilidad desde dispositivos móviles.
\item \textbf{Añadir más funcionalidades a la aplicación:} Añadir más gráficas al estudio, para obtener un análisis más completo. También se puede mejorar el formulario de búsqueda ofreciendo a los usuarios el resultado de las búsquedas de páginas en Facebook, eliminando el paso de comprobación del nombre y de la existencia de la sección \textit{Post To Page}. 
\item \textbf{Definir mejor los resultados realizando análisis de sentimiento:} Incluir análisis de texto para identificar y extraer información subjetiva de los comentarios de los usuarios, este procesado de los datos intenta determinar la actitud de un usuario ante un suceso. Con esta mejora se podrían calificar los comentarios como positivos o negativos, pudiendo deducir si son quejas o expresan gratitud.  
\item \textbf{Mejorar la concurrencia del sistema:} Dividir en hilos las consultas a la API de Facebook, de forma que se reduzca el tiempo de espera del usuario.
\end{itemize}